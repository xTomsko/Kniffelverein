%!TEX root = ../../dokumentation.tex
\section{Aufgabenstellung (Alex)}
Im Rahmen der vorliegenden Studienarbeit soll ein entsprechendes System entwickelt werden mit dessen Hilfe analoge papierbasierte Spielergebnisse des
Gesellschaftsspieles „Kniffel“ beziehunsgweise „Yathzee“ digitalisiert werden können. Für die Erkennung der einzelnen Spielergebnisse und ide Überführung in ein digitales 
Format soll eine automatische \textit{Machine Learning}-gestützte Bilderkennung eingesetzt werden. Nach der Digitalisierung der Daten, sollen die Spielergebnisse
ausgewertet und visualisiert werden.

Für die Implementierung der Webapplikation können drei große Arbeitspakete identifiziert werden. Zuest müssen die Daten hochgeladen werden können und das hochgeladene
Bild mit den analogen Spielergebnissen für die Bildverarbeitung vorbereitet werden. Nach dem Prozess des Image Processing muss mit Hilfe eines \textit{Machine Learning}-
Algorithmus die einzelnen Ziffern auf dem Bild erkannt werden. Für die Bilderkennung soll ebenfalls wissenschaftlich untersucht werden, welcher Algorithmus am 
geeignetsten für das Projekt ist. Dafür werden die drei \textit{Machine Learning}-Algorithmen \textit{K-Nearest-Neighbor}, die \textit{Support Vector Machine} und
\textit{Convolutional Neural Networks} miteinander verglichen werden.
Nach der Bilderkennung sollen die Spielergebnisse dann ausgewertet und für den Nutzer visualisiert werden.

Für das Software-Projekt wurden die User Stories in der nachfolgenden Tabelle  \ref{table:user_stories} definiert.
\pagebreak
\begin{table}[]
    \begin{tabular}{m{2.5cm}|m{12cm}}
    \textbf{ID} & \textbf{User Story}                                                                                              \\ \hline
    01          & Als ein Nutzer SOLL ein Bild hochgeladen werden können                                                           \\
    02          & Als ein Nutzer SOLLEN die Spielergebnisse auf dem Bild mit einem Machine Learning Algortihmus ausgewertet werden \\
    03          & Als ein Nutzer SOLLEN die digitalisierten Spielergebnisse überprüfbar und wenn nötig korrigierbar sein           \\
    04          & Als ein Nutzer SOLLEN die digitalisierten Spielergebnisse in einer Datenbank abgespeichert werden                \\
    05          & Als ein Nutzer SOLL eine statistische Auswertung aller hochgeladenen Spielergebnisse zur Verfügung stehen        \\
    06          & Als ein Nutzer KANN man sich registrieren                                                                        \\
    07          & Als ein Nutzer KANN man sich ein- und ausloggen                                                                  \\
    08          & Als ein Nutzer KANN man seine eigenen hochgeladenen Spielergebnisse anzeigen lassen                             
    \end{tabular}
    \caption{User Stories}
    \label{table:user_stories}
\end{table}
