%!TEX root = ../dokumentation.tex
\chapter{Ergebnisse und Ausblick} \label{cha:results}
Im Rahmen der Arbeit ist eine Anwendung zur Auswertung eines Kniffel Ergebnisblatt unter Verwendung eines Machine Learning Algorithmus erstellt worden. Die Anwendung besteht aus einem Webserver, welcher die Schnittstelle zwischen dem Benutzer und der Verarbeitung darstellt. Der Webserver ermöglicht den Upload eines Ergebnisblattes und startet danach dessen Verarbeitung. Des Weiteren ermöglicht der Webserver einen Einblick in die Statistik der gespeicherten Daten. Die Ergebnisse eines neu ausgewerteten Blattes werden dabei mit abgespeichert und halten die Statistiken auf einem aktuellen Stand.\\\hfill
Die Webseiten wurden ansprechend und schlicht gestaltet, sodass eine einfache Bedienung ermöglicht wird. Ein weiterer positiver Punkt für eine einfache Bedienung ist die Möglichkeit, dass ein Bild auch dann verarbeitet werden kann, wenn es nicht perfekt gerade fotografiert wurde. Auf diese Weise kann schnell ein Bild gemacht werden ohne viel zu beachten.\\\hfill
Jedoch bedeutet dies auch eine große Einschränkung, denn das \textit{Image Alignment} ist nur mit dem vorgegebenen Ergebnisblatt möglich.\\\hfill
Die Erkennung der Zahlen ist dafür gestaltet, um mit ein-, zwei- und dreistelligen Zahlen klarzukommen. Der \textit{CNN-Algorithmus} bietet mit 99,16\% zwar nicht die höchste Genauigkeit, im Vergleich mit den anderen Algorithmen, allerdings ist die Bildgröße von 28×28 Pixeln anstatt der 8×8 Pixel ein großer Vorteil. Dieser Vorteil bietet auch mehr Sicherheit bei der Erkennung.\\\hfill
Die Statistiken bieten einen guten Einblick der vergangenen Spiele und stellt die erreichten Punkte übersichtlich dar. Die einzelnen Graphen bieten einen guten und einfachen Überblick des Verlaufs und ist damit eine um einiges bessere und angenehmere Möglichkeit als das Vergleichen der einzelnen Blätter miteinander.\par

Trotz aller Genauigkeit beim Erkennen der Zahlen in den einzelnen Zellen, kann dieser Punkt noch weiter verbessert werden. So sollte sichergestellt werden, dass Konturen nicht falsch erkannt werden, z.B. wenn zwei Striche einer Zahl etwas zu weit voneinander entfernt sind. Des Weiteren sollte auch eine Möglichkeit für den Nutzer eingeführt werden, um den Algorithmus zu kontrollieren. Dabei muss aber auch bedacht werden, dass eine Kontrolle nicht verpflichtend sein sollte.\\\hfill
Aber auch beim Upload der Bilder können Verbesserungen vorgenommen werden. Dazu gehört die Verarbeitung von allen möglichen Kniffel Ergebnisblättern zu realisieren und es nicht nur auf die Vorlage zu beschränken.\\\hfill
Desweiteren kann man über ein Speichern der Bilder nachdenken, um die Möglichkeit zu bieten auch im Nachhinein nochmal darüber zu schauen und, sollte es implementiert sein, nochmal zu kontrollieren.\\\hfill
Als Letztes kann noch eine Art Benutzerverwaltung eingeführt werden und somit Statistiken für mehrere Nutzer anbieten. Die zusätzlichen Nutzer bieten dann auch die Möglichkeit für zusätzliche Statistiken, dazu gehört einem Vergleich der Leistungen der Benutzer untereinander.
