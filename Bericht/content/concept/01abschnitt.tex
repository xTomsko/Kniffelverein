%!TEX root = ../../dokumentation.tex
%Konzept
\section{Genutzte Technologien (Alex)}
In diesem Kapitel werden die verwendeten Technologien und Frameworks beschrieben, die für das Projekt benötigt werden.
Für die Umsetzung des Projektes fiel die Wahl auf die Programmiersprache \textbf{Python}. Sie ist vielseitig einsetzbar und es existieren bereits Bibliotheken mit
Funktionalitäten, die in diesem Projekt benötigt werden. Diese relevanten Bibliotheken werden in den nächsten Unterkapiteln näher beleuchtet.
Da wäre zum einen, die freie Programmbibliothek \textbf{OpenCV}, die den Bereich Bildbearbeitung abdeckt. Ebenfalls relevant ist das \textbf{SciKit-learn}, das 
bereits implementierte \textit{Machine Learning}-Algorithmen zur Benutzung anbietet. Außerdem werden für das Projekt das Webframework \textbf{Flask} sowie
die Bibliothek \textbf{Dash} genutzt. Beschrieben werden nur die relevanten Informationen, die für das Verständnis dieser Arbeit wichtig sind. 
Es besteht kein Anspruch auf Vollständigkeit in diesem Kapitel.

\subsection{OpenCV}
\textbf{OpenCV} ist eine plattformunabhängige Bibliothek im Bereich Computer Vision. Sie stellt Funktionen für verschiedenste Prozesse der Bildverarbeitung bereit.
Dazu gehören unter anderem Bildanalyse und Objekterkennung, Kamerakalibrierung und Stereo-Vision sowie Ansätze für Machine Learning und Robotics.
Die quelloffene Bibliothek wurde 1999 veröffentlicht und wird weltweit durch Entwickler in Forschungseinrichtungen und Unternehmen weiterentwickelt. 
Eine kommerzielle Nutzung ist aufgrund der BSD-Lizenz möglich. Die Implementierungen in \textbf{OpenCV} sind für echtzeitfähige Anwendungen konzipiert 
und dementsprechend optimiert. 
Teilweise existieren zusätzliche Implementierungen die mit CUDA oder OpenCL unter Nutzung der GPU beschleunigt werden.
Durch sein breites Anwendungsspektrum ist \textbf{OpenCV} weitverbreitet und repräsentiert für viele Bereiche den Stand der Technik. 
\textbf{OpenCV} beinhaltet einen ORB Feature Detector und Descriptor sowie eine Sammlung von Algorithmen zur Zuordnung von Features.
Diese Funktionalität kann bei der Bildvorverarbeitung der eingescannten Spielergebnisse verwendet werden.

\subsection{SciKit-learn}
\textbf{SciKit-learn} ist eine freie Softwarebibliothek, die \textit{Machine Learning}-Algorithmen für die Programmiersprache Python bereitstellt. 
Dazu gehören verschiedene Klassifikations-, Regressions- und Clustering-Algorithmen, wie zum Beispiel die \textit{Support Vektor Machine}, 
Random Forest, Gradient Boosting, k-means oder DBSCAN. Sie basiert als SciKit (Kurzform für SciPy Toolkit)
auf den numerischen und wissenschaftlichen Python-Bibliotheken \textbf{NumPy} und \textbf{SciPy}.
Sie ist in verschiedenen Anwendungsbereichen einsetzbar und bietet effiziente Tools für maschinelles Lernen, Data Mining und Datenanalyse an.
Weitere Funktionalitäten die \textbf{SciKit-learn} für das maschinelle Lernen anbietet, sind Clustering, Cross Validation, Reduktion, Feature-Extraktion und Parameter-Tuning.
Außerdem wird das Zusammenfassen und Darstellen der Daten vereinfacht.

Durch \textbf{SciKit-learn} konnten sehr einfach die einzelnen \textit{Machine Learning}-Algorithmen \textit{K-Nearest Neighbors}, \textit{Support Vector Machine} und
\textit{Convolutional Neural Networks} ausprobiert werden. Es brauchte somit keine eigene Implementierung der Algorithmen und sie konnten ohne großen Aufwand
miteinander verglichen werden. Auch bei dir Implementierung der eigentlichen Ziffernerkennung kann das \textbf{SciKit-learn} eingesetzt werden.

\subsection{Flask}
\textbf{Flask} ist Webframework für \textbf{Python}, das vom österreichischen Programmierer Armin Ronacher entwickelt worden ist.
Bei diesem minimalistisch gehaltenem Framework liegt der Fokus auf eine hohe Erweiterbarkeit durch weitere Module und eine gute Dokumentation.
Es bestehen nur zwei Abhängigkeiten zu weiteren Bibliotheken. Das wären zum einen die Template-Engine Jinja2 und Werkzeug, eine Bibliothek zur Erstellung
von WSGI-Anwendungen. \textbf{Flask} nutzt für die Kommunikation zwischen Webserver und Webanwendung die Python-WSGI-Schnittstelle (Web Server Gateway Interface).
Um die Programmierung zu erleichtern, stellt \textbf{Flask} für Testzwecke einen mitgelieferten Webserver bereit. Ein weiteres Unterscheidungsmerkmal von \textbf{Flask}
im Vergleich zu den anderen \textbf{Python}-Webframework, wie Django oder Web2py, ist, dass keine eigenen Komponenten angeboten werden, sondern die Funktionalitäten können
sehr einfach durch weitere bestehende Bibliotheken erweitert werden. Das trägt dazu bei, dass \textbf{Flask} sehr minimalistisch gehalten werden konnte.
Für die wichtigsten Funktionalitäten existieren bereits Erweiterungen, wie zum Beispiel für das Handhaben von Authentifizierung, Cookies und Sessions, das Konfigurieren
des Cachings oder die Kompatibilität zu vielen Datenbanksystemen, wie MySQL, PostgreSQL oder MongoDB. 
Deshalb wird in diesem Projekt das Webframework \textbf{Flask} verwendet.

\subsection{Dash}
Plotly ist ein Unternehmen für technische Datenverarbeitung mit Hauptsitz in Montreal, Kanada, das Online-Tools für Datenanalyse sowie Datenvisualisierung entwickelt. 
Plotly bietet Online-Grafik-, Analyse- und Statistik-Tools sowie wissenschaftliche Grafikbibliotheken für Python, Perl, Arduino und weitere an.
\textbf{Dash} ist ein Open-Source Python-Framework für die Entwicklung webbasierter Analyseanwendungen und Dashboards. 
Außerdem gibt es viele spezialisierte Open-Source-Bibliotheken für \textbf{Dash}, die auf die Entwicklung spezifischer \textbf{Dash}-Komponenten und 
Anwendungen zugeschnitten sind. Das \textbf{Dash}-Framework wird in diesem Projekt für die nachträgliche Analyse und Visualisierung der digitalisierten
Spielergebnisse verwendet.
