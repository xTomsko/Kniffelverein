%!TEX root = ../../dokumentation.tex
%Theorie
\section{Machine Learning}
In den nachfolgenden Kapiteln werden die nötigen Grundlagen zum Thema \textit{Machine Learning} erläutert, die für das Verständnis dieser Arbeit wichtig sind.
Zu Beginn wird es eine kurze Einleitung geben, in der \textit{Machine Learning} definiert und erklärt wird. Außerdem wird die MNIST-Datenbank vorgestellt, deren
Daten als Basis für unsere Algorithmen dienen wird. Daraufhin werden die einzelnen \textit{Machine Learning}-Algorithmen näher beschrieben, die in die engere
Auswahl für das Projekt gekommen sind. Dazu gehören die Algorithmen \textit{K-Nearest Neighbor}, die \textit{Support Vector Machine} und \textit{Convolutional
Neural Networks}. Diese werden dann in einem späteren Kapitel miteinander verglichen, um den am besten geeigneten \textit{Machine Learning}-Algorithmus 
für dieses Projekt zu ermitteln.

\subsection{Grundlegende Theorie zu \textit{Machine Learning}}
\textit{Machine Learning} oder auf Deutsch Maschinelles Lernen ist ein Teilgebiet innerhalb der Informatik, das sich aus dem Bereich der Künstlichen Intelligenz
entwickelt hat. Es handelt sich hierbei um die Entwicklung von Algorithmen, die auf Basis eines Datensatzes trainiert werden und daraufhin Vorhersagen über
weitere unbekannte Daten treffen kann. Beim Maschinellen Lernen wird im Gegensatz zur traditionallen Programmierung keine Liste an statischen Programmanweisungen
ausgeführt, sondern es wird anhand von Beispieldaten ein Modell konstruiert, mit dem dann datengesteuerte Vorhersagen oder Entscheidungen getroffen werden
können.\cite{simon_2015}

Wenn wir zum Beispiel mit Hilfe eines Programmes den Spritverbrauch eines PKWs vorhersagen wollen, müssen zuerst Daten über das Auto gesammelt werden, die womöglich
in Zusammenhang mit dem Spritverbrauch stehen könnten. Diese Daten des PKWs werden dann mit dem dazugehörigen Spritverbrauch in einen der verschiedenen verfügbaren 
\textit{Machine Learning}-Algorithmen gegeben, der auf Basis dieser Daten trainiert wird. 
Nachdem der Prozess des Maschinellen Lernens abgeschlossen ist, können auch Parameter bis dahin unbekannter Autos hineingegeben werden und der Algorithmus kann nun
anhand der Traingsdaten eine Prognose für den Spritverbrauch des unbekannten Autos abgeben.
\textit{Machine Learning} kann in vielen unterschiedlichen Situationen eingesetzt werden. Es braucht jedoch immer eine große Anzahl an Traingsdaten, um einen
solchen Algorithmus sinnvoll trainieren zu können.

Eine einfache Form des \textit{Machine Learning} ist die Lineare Regression. Dabei handelt es sich auch um einen Algorithmus, der auch Vorhersagen treffen kann.
Die Lineare Regression basiert auf einer mathematischen Gleichung der Form: $h(x) = \theta_0 + \theta_1x $ mit $\theta_0$ und $\theta_1$ als Konstanten.
Die Trainigsdaten enthalten jeweils einen Input $x$ und einen dazugehörigen Output $y$. Bei dem obigen Beispiel wäre $y$ der Spritverbrauch und $x$ zum Beispiel der
Hubraum. Beim Trainieren versucht der Algorithmus die beiden Konstanten $\theta_0$ und $\theta_1$ in der Gleichung $h(x)$ so zu wählen, 
dass der Unterschied zum erwarteten Output $y$ minimiert wird. Nach dem Trainieren ist die Gleichung optimiert und es wurden Werte für die beiden Konstanten ausgewählt.
Nun kann auch ein unbekanntes $x$ in die Funktion gegeben werden, die dann eine Vorhersage für $y$ berechnet.\cite{simon_2015}

Das wäre nur ein einfaches Beispiel für den Einsatz von Maschinellem Lernen. Es können jedoch auch mathematisch kompliziertere Sachverhalte durch \textit{Machine Learning}
gelöst werden. Ein Einsatzbereich ist zum Beispiel die Erkennung von handgeschriebenen Ziffern anhand einer Bilddatei. Auch hier kann durch Maschinelles Lernen ein
Algorithmus mit einer Vielzahl an Bilddateien trainiert werden, damit dieser enthaltene Muster erkennt und eine Vermutung für die abgebildete Ziffer abgibt. 

\subsection{MNIST}
Für die Erkennung von handgeschriebenen Ziffern trainieren wir den \textit{Machine Learning}-Algorithmus mit den Daten aus der MNIST-Datenbank.
MNIST steht für Modified National Institute of Standards and Technology Datenbank. 
Es ist eine öffenltich zugängliche Datenbank, die handgeschriebene Ziffern enthält.
Sie enthält einen Trainigssatz aus 60.000 Beispielen, sowie einen dazugehörigen Testdatensatz, der 10.000 Testobjekte enthält.
Dabei ist jede Ziffer als ein Graustufen-Bild der Größe 28 x 28 abgespeichert.
Mit diesem Datensatz können sehr einfach \textit{Machine Learning}-Algorithmen, wie \textit{Convolutional Neural Networks} oder eine \textit{Support Vector Machine} trainiert
werden, um dann unbekannte handschriftliche Ziffern zu erkennen.
