\section{Vergleich verschiedener \textit{Machine Learning}-Algorithmen} \label{sec:compare}
In diesem Kapitel sollen unterschiedliche \textit{Machine Learning}-Algorithmen im Hinblick auf das Projekt miteinander verglichen werden.
Es wurde bereits eine Vorauswahl getroffen und für den Vergleich wurden die Algorithmen \textit{K-Nearest-Neighbor}, die \textit{Support Vector Machine} und
\textit{Convolutional Neural Networks} ausgewählt. Für einen fairen Vergleich wird beim Trainieren aller zuvor genannten \textit{Machine Learning}-Algorithmen
die bereits vorgestellte MNIST-Datenbank verwendet, die Datensätze zu handschriftlichen Ziffern enthält.

Die Kriterien, nach denen der Vergleich durchgeführt werden soll, wurden bereits im Vorfeld definiert. Es wurde sich auf die Kriterien Komplexität,
Präzision, Genauigkeit sowie die Einfachheit der Umsetzung geeinigt. Die \textbf{Komplexität} beschreibt hier die \textit{(Zeit-)}\textbf{Komplexität} des verwendeten Algorithmus und wird mit der O-Notation dargestellt.
Die \textbf{Genauigkeit} bei \textit{Machine Learning}-Algorithmen ist definiert durch die Anzahl der richtig zugewiesenen Trainingsobjekte durch die Anzahl aller
Trainingsobjekte, während die \textbf{Präzision} durch folgende Gleichung bestimmt werden kann:
\[ \text{Präzision} = \frac{\text{Richtig Positiv}}{\text{Richtig Positiv} + \text{Falsch Positiv}} \]
Danach wird die \textbf{Einfachheit der Umsetzung} der zu bewertenden \textit{Machine Learning}-Algorithmen von der testenden Person auf Basis einer Skala von
eins bis fünf beurteilt.

In den folgenden Unterkapiteln werden die drei \textit{Machine Learning}-Algorithmen aus der Vorauswahl näher beschrieben und eine Evaluation nach den genannten
Kriterien vorgenommen.


