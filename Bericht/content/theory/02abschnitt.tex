\section{Vergleich verschiedener Machine Learning Algorithmen}
Wir nehmenen die Daten von MNIST
\subsection{KNN}

\subsection{SVM}

\subsection{CNN}
Die Convolutional Neural Networks (CNN) sind vergleichbar zu den traditionellen Neuronalen
Netzwerken. Im Aufbau ähneln sie sich darin, dass beide Modelle aus Neuronen mit Gewichten 
bestehen, die ein Skalarprodukt aus einer Eingabe bilden. Das CNN besitzt ebenfalss eine
Verlustfunktion. Ein typisches Neuronales Netzwerk erhält als Eingabe einen einzigen Vektor und
wird in verscheidenen Schichten verarbeitet. 
Jede Schicht wird aus Gruppen von Neuronen gebildet, die jeweils mit den Neuronen aus der vorherigen
Schicht verknüpft sind. Die Neuronen, die sich in der selben Schicht befinden, teilen keine Verküpfungen und sind
unabhängig von den anderen.

Dabei existieren drei Hauptarten von Schichten in einem gewöhnlichen CNN, die Filter-Schichten (engl. Convolutional Layer), Aggregations-Schichten (Pooling Layer) 
und den vollständig verbundenen Schichten (engl. Fully Connected Layer, Dense Layer). 
Zusätzlich gibt es eine Eingabeschicht und eine Ausgabeschicht. Die Ausgabeschicht stellt dabei eine vollständig verknüpfte Schicht dar.

\subsubsection{Input Layer}
In der Eingabeschicht werden die Bilddaten gespeichert und in einer dreidimensionlae Matrix dargestellt.

\subsubsection{Filter-Schicht}
Die Filterschicht, auch häufig als Feature-Extraction-Layer bezeichnet, extrahiert Eigenschaften aus dem Eingabebild und übt die Hauptanzahl an Berechnungen aus. 
Es handelt sich hierbei um Faltungsoperationen, die Ähnlichekeiten zur Fourier-Transformation und zur Lapace-Transformation aufweisen, und bilden das Merkmal eines CNNs.
Ein Neuron einer Filterschicht betrachet ein bestimmten Bereich einer vorherigen Schicht in Form einer Matrix und bildet daraus ein Skalarprodukt, um die den Bereich auf nur eine Zahl zu reduzieren.
Die Architektur eines CNNs, die aus den verschiednenen Schichten und der Filter-Schichten besteht, ermöglichen, dass weniger Neuronen benötigt werden im Gegensatz zu anderen Mulitlayer Neuronalen Netzwerken.

\subsubsection{Aggregations-Schicht}


\subsection{Auswertung des Vergleiches}
