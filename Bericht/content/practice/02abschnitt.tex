%!TEX root = ../../dokumentation.tex
%Umsetzung
\section{Zahlenerkennung} \label{sec:digit_rec}
Durch die in der Bildvorverarbeitung \ref{sec:imagepreprocessing} vorgenommenen Schritte wird das gesamte Blatt in die einzelnen Zellen aufgeteilt. Die Zellen werden dabei Spaltenweise bearbeitet, um so immer einen Spieldurchgang nach dem anderen abzuarbeiten. Durch die Aufteilung des kompletten Bildes in die einzelnen Zellen ist es möglich jeweils nur auf einen kleinen Teil des Bildes zu konzentrieren.\\\hfill
Um die einzelnen Zellen an den Algorithmus übergeben zu können und die erkannte und hoffentlich richtige Zahl zu erhalten, müssen zunächst ein paar Schritte unternommen werden. Diese Schritte werden dabei immer mit dem ursprünglichen Bildausschnitt, als der einen Zelle, durchgeführt. Als allererstes wird zunächst das Bild von einem farbigen in ein Graustufenbild gewandelt. Das geschieht, um die Farben zu entfernen, da diese den gesamten Prozess übermäßig kompliziert machen und für diese Klassifizierung keinen Mehrwert haben. Da es sich in fast jeder Zelle mehr als eine Zahl befinden kann, muss man zunächst eine Kantenerkennung auf das Bild anwenden. Hierfür müssen wir den Noise, also kleine Flecken oder andere störende Dinge, die zu klein für eine Zahl sind, entfernen. Jetzt können wir die Funktion \textit{findContours} von \textit{OpenCV} verwenden und erhalten damit die Anzahl der Konturen sowie deren Koordinaten. Diese Koordinaten teilen das Bild der ursprünglichen Zelle weiter auf, sodass nur noch eine Zahl im Fokus ist. Um mit diesem neuen Bildausschnitt eine möglichst gute Erkennung zu gewährleisten, wird das Bild in ein sogenanntes Binärbild umgewandelt. Das heißt, dass das Bild danach nur noch aus schwarzen und weißen Pixeln besteht. Um dies zu erreichen, wird eine Grenzwertfunktion auf jedes Pixel angewendet. Ist der Grauwert des Pixels über dem Grenzwert wird er auf Weiß gesetzt, fällt er jedoch unter den Grenzwert wird der Pixel auf Schwarz gesetzt.\par
Durch diese Schritte haben wir uns für jede Zelle und jede darin enthaltende Kontur einen Bildausschnitt erzeugt und ihn in ein Binärbild gewandelt. Da wir unseren Algorithmus mit einer bestimmten Bildgröße trainiert haben, müssen wir den Bildausschnitt auch noch auf diese Größe anpassen. Dafür können wir, sollte der Ausschnitt zu groß sein, die \textit{Resize} Funktion von \textit{OpenCV} verwenden. Diese Funktion könnten wir auch verwenden, um ein kleines Bild zu vergrößern, allerdings ist es besser statdessen das Bild mit schwarzen Pixeln aufzufüllen. Die zusätzlichen Pixel befinden sich ausschließlich am Rand des Bildausschnittes und erfüllen lediglich die Größenanforderungen des Algorithmus.\par
Nachdem wir nun einen Bildausschnitt mit einer Zahl haben, der sich nun auch in der richtigen Größe befindet, können wir ihn dem Algorithmus zur Klassifizierung übergeben. Die an dieser Stelle offensichtlich gewordenen Probleme werden im Kapitel \ref{sec:problem_number_detection} erläutert.\\\hfill
Der \textit{CNN} gibt eine Liste von Wahrscheinlichkeiten für jede der Zahlen, also die Zahlen 0–9, zurück. Diese Wahrscheinlichkeit gibt an, wie sicher sich der Algorithmus ist, dass es sich um diese Zahl handelt. Da es automatisch abgearbeitet wird und nicht jede Zahl kontrolliert wird, wird immer die Zahl mit der höchsten Wahrscheinlichkeit genommen.\\\hfill
Sollte in der Zelle mehr als eine Kontur sein, also eine zwei- oder dreistellige Zahl darin stehen, wird dasselbe Verfahren auch für jede weitere Kontur angewendet. Die auf diese Art erhaltenen Zahlen werden nun anhand ihrer Position im Bild zu einer zweistelligen oder eben dreistelligen Zahl kombiniert.\par
Sobald dieser Prozess für jede Zelle durchgeführt wurde, werden die Zahlen mit den vorhandenen Informationen zur Zelle in einer Datenbank gespeichert, sodass mit der Visualisierung \ref{sec:dashboard} fortgefahren werden kann.
