%!TEX root = ../dokumentation.tex

\pagestyle{empty}

\chapter*{Kurzfassung}	% deutsch
Maschinelles Lernen ist seit einigen Jahren ein wichtiger Bereich in der Informationstechnik und dort nicht mehr wegzudenken. Die Einsatzmöglichkeiten sind dabei nahezu unbegrenzt. So wird es von Netflix, oder anderen Streaminganbietern, zum Vorschlagen neuer Serien und Filme genutzt, als Spamfilter bei E-Mails oder digitale Assistenten wie \textit{Alexa} von Amazon oder \textit{Google Now} von Google. Aber auch zur Objekterkennung in Bilder wie in der Medizin zur Erkennung von Tumoren oder zur Gesichtserkennung bei Kameras. Die Entwicklung von maschinellem Lernen ist dabei keineswegs abgeschlossen, sondern schreitet immer weiter voran. Es werden somit immer neue Einsatzgebiete gefunden, dazu gehört z.B. das autonome Fahren.\par

Im Rahmen dieser Arbeit sollen die Technologien des maschinellen Lernens verwendet werden, um ein Kniffel Ergebnisblatt auszuwerten. Die ausgewerteten Ergebnisse sollen ansprechend dargestellt werden, um dem Nutzer einen Überblick zu ermöglichen. Außerdem soll eine langfristige Auswertung der Ergebnisse ermöglicht werden.\\ \hfill
Für ein möglichst gutes und genaues Ergebnis soll zunächst ein Vergleich zwischen verschiedenen Machine Learning Algorithmen durchgeführt werden. Im Zuge dieser Arbeit wurde sich dabei für die Algorithmen \textit{K-Nearest-Neighbor (KNN)}, \textit{Support Vector Machine (SVM)} und \textit{Convolutional Neural Network (CNN)} entschieden.


\newpage
\chapter*{Abstract} % englisch
Machine learning has been an important area in information technology for several years and it is impossible to imagine life without it. The possible applications are almost unlimited. For example, it is used by Netflix or other streaming providers to suggest new series and movies, as a spam filter for e-mails or digital assistants such as \textit{Alexa} from Amazon or \textit{Google Now} from Google. But also for object recognition in images like in medicine for tumor detection or for face recognition in cameras. The development of machine learning is by no means finished, but continues to progress. Thus, new fields of application are found all the time, including autonomous driving, for example.\par

In the context of this work the technologies of the machine learning are to be used, in order to evaluate a Kniffel result sheet. The evaluated results shall be presented in an appealing way to provide the user with an overview. Furthermore, a long-term evaluation of the results shall be made possible.\\\hfill
For the best and most accurate results, a comparison between different machine learning algorithms should be performed first. In the course of this work, the algorithms \textit{K-Nearest-Neighbor (KNN)}, \textit{Support Vector Machine (SVM)} and \textit{Convolutional Neural Network (CNN)} were chosen.
